\documentclass{article}

\usepackage{minted}
\usepackage{forest}

\pagenumbering{gobble}

\begin{document}

{\Large \textbf{Reference C++ Code A}}
\begin{minted}{c}
    if (fork() == 0) {
        // do stuff
    } else if (fork() == 0) {
        // do more stuff..
    } else if (fork() == 0) {
        // do more "stuff"..
    } else if (fork() == 0) {
        // noooo
    }
\end{minted}

{\Large \textbf{Reference C++ Code B}}
\begin{minted}{c}
    if (fork() == 0) {
        if (fork() == 0) {
            if (fork() == 0) {
                // if ...
                // aaaahhhh
            }
        }
    }
\end{minted}

{\Large \textbf{Reference C++ Code C}}
\begin{minted}{cpp}
    int main(int argc, char* argv[]) {
        // just use execv to run ANY program
        if (execv("/usr/bin/gedit", argv) == -1) {
            // error
            cout << "Error. Booooo!" << endl;
        }
        cout << "Will this line still be printed?" << endl;
    }
\end{minted}

These code snippets were obtained from the Lab 6 assignment page.

\vfill
\newpage

\begin{enumerate}
    {\large \item What should be the resulting process \emph{FAMILY TREES} from these two (Code A and Code B) code snippets? Illustrate.}
    
    In reference code A all \texttt{fork()} calls performed at the same "hierarchy level" in the if-else tree. The entry point starts at \texttt{if (fork() == 0)},
    which produces the first child process. However, should this \texttt{fork()} fail, the succeeding \texttt{fork()} calls in the following \texttt{else if (fork() == 0)}
    statement produces a child process accordingly. Once a \texttt{fork()} successfully produces a child, the program enters the respective instruction block and stops forking.
    \textbf{The resulting process family tree from code A looks like:}
    \vspace{1em}

    \begin{forest}
        [Parent
            [Possible Child 1]
            [Possible Child 2]
            [Possible Child 3]
            [Possible Child 4]
        ]
    \end{forest}

    \emph{In this diagram, Possible Child $n$ for $n \geq 2$, only gets instantiated if the first $n - 1$ fork(s) fail.}

    In reference code B \texttt{fork()} calls are performed sequentially, with further calls only being called if the previous one succeeds. If the first \texttt{fork()}
    succeeds, then the second one is called, then if this succeeds the third one is called; so on and so forth. At the first \texttt{fork()} the parent produces a child,
    in the second one both parent and child produce a child, at the third call the parent, child, and the \emph{grandchild} all produce a child---effectively at each call
    the amount of total processes \emph{doubles}, demonstrating exponential growth. \textbf{The resulting process family tree from code B, can be illustrated
    per stage (up to the third fork):}
    \vspace{2em}

    \begin{center}
        {\large \textbf{1st Fork}}

        \begin{forest}
            [Parent
                [Child 1*]
            ]
        \end{forest}
    \end{center}
    \vspace{5em}

    \begin{center}
        {\large \textbf{2nd Fork}}

        \begin{forest}
            [Parent
                [Child 2**]
                [Child 1*
                    [Child 3**]]
            ]
        \end{forest}
    \end{center}
    \vfill
    \newpage

    \begin{center}
        {\large \textbf{3rd Fork}}

        \begin{forest}
            [Parent
                [Child 4***]
                [Child 2**
                    [Child 5***]]
                [Child 1*
                    [Child 3**
                        [Child 7***]]
                    [Child 6***]]
            ]
        \end{forest}
    \end{center}

    \emph{In this diagram each * denotes the $n$th fork at which the child was instantiated, where $n$
    represents the amount of asterisks trailing a child label.}

    Effectively at the third fork there will be a total of \textbf{8} processes, since at each fork
    in this structure the amount of processes double, we can express this as $2^n$ where $n$ is the
    amount of times the program calls \texttt{fork()}.


    {\large \item Will the last line in the sample code below still be printed? How about when using \texttt{execl()}? Why or why not? Explain.}
\end{enumerate}

\end{document}
\documentclass{article}

\usepackage[a4paper, margin=1.05in]{geometry}

\begin{document}
\centering
{\huge \textbf{Intro to Assembly I: GAS \& AT\&T Syntax}}
\vspace{0.5cm}

\begin{enumerate}
    \item \textbf{(Max) Given an integer \( x \), can \( x \) multiply by 19 be implemented by using shifts and adds only? How?}
    
    First we define "\( x \) multiply by 19" as the equation:
    \[
    x \cdot 19 = 19x
    \]
    Notice that 19 can be broken into a sum of 16 and 3 which allows us to rewrite the above to:
    \[
    19x = (16+3)x 
    \]
    \[
    19x = 16x + 3x
    \]
    Bit shifting manipulates numbers through powers of 2, we want to break down the expression such that we
    can express it as the sum of powers of 2. By inspection we can see that 16 is a power of 2. Using the same
    technique above we can also break down 3.
    \[
    19x = 2^4x + (2+1)x
    \]
    \[
    19x = 2^4x + 2^1x + x
    \]
    Now that we're able to represent the product as a sum of powers of 2, we can then substitute the powers of 2 to
    their corresponding bit shifts.
    \[
    19x = (x << 4) + (x << 1) + x
    \]

    \item \textbf{(Paco) What does the assembly version do? Does it use the multiply instruction?}
    
    your mom is a placeholder

    \item \textbf{(Paco) What happens for the case of \( x \cdot 45 \)?}
    
    your mom is a placeholder

    \item \textbf{(Ian) What happens for the case of \( x \cdot -2 \)?}
    
    your mom is a placeholder

    \item \textbf{(Max) What happens for the case of \( x \cdot  0 \)?}
    
    your mom is a placeholder
\end{enumerate}

\end{document}